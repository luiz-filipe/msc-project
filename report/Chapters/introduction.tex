\chapter{Introduction}
\label{ch:introduction}

Recent studies have estimated that there are $8.7$ million eukaryote species on the planet \cite{10.1371/journal.pbio.1001127} \cite{10.1371/journal.pbio.1001130}. Most of them are still to be discovered. \citeauthor{dawkins1990selfish} \cite{dawkins1990selfish} argues that the ultimate goal of every single one of them is survival, and to achieve that they do extraordinary things to solve the most varied range of challenges that the environment they live in imposes on them. Birds migrate thousands of miles in search of food, fish spend most of their  energy swimming up against the stream in order to reach the perfect location to lay their eggs. One could argue that these animals are intelligent, but here are numerous ways to define intelligence. \citeauthor{kennedy2001swarm} \cite{kennedy2001swarm} is part of the group of people who believe that intelligence is mostly about the capacity to adapt to new situations that have not been foreseen  \cite{kennedy2001swarm}, and one shares this same point of view.

We are surrounded by examples of animals with a wide variety of complexity, ranging from simple organisms with limited intelligence to us humans beings. All meet the challenged of surviving in an ever changing environment in  different and ingenious ways. It could be argued that humans are the most successful creatures on this planet. It is very easy to blindly accept this statement, for the vast majority of us grow up being told that is the case. Nevertheless, there are different ways to think about 'success'. If survival is all that matters, the number of individuals in a certain species could be used as a good benchmark for success, and in this case humans are losing emphatically to insects \cite{MAY16091988}.

So how can these, as far as we are concerned, 'limited' creatures accomplish very complex tasks? How can they react to the highly dynamic environments they live in? For example, when worker ants leave the nest to forage in the morning, the environment around the nest can be completely different from what it was last time they went out. Leaves might have fallen, thus changing the route to access food in the rain forest; the wind can blow the sand in the desert, covering food sources. Yet despite these unpredictable factors, the ants face the necessity to locate food, and indeed they do. As a complex decentralised system, the colony agents are able to overcome the disruptions introduced by the environment and keep going on, meeting the challenge of survival. These questions are very hard to answer, and indeed this project has no intention to do so. 
 
Systems of social insects are difficult to study due to the very nature of the structures they create. These structures are decentralised and behaviours emerge from interactions between the system's agents - the individual insects of which there are a great number. Accordingly, understanding of the agents' behaviour in isolation does not guarantee and understanding of the overall picture. Also, small changes in the way agents interact and react to external stimuli may introduce large changes in the resulting overall behaviour; the complexity of these interactions and the amount of possible variables that can affect the system as a whole are so large that we easily fail to appreciate the incredible achievement that these complex systems are by themselves.

This project has as objective to study how agents and the ant colony as a whole react to changes in some properties of the pheromones that the ants use to communicate indirectly to each other. In simple terms, ants use different pheromones to transmit different types of messages. As chemical substances, pheromones interact with the environment differently from one another. As a byproduct, agents are affected in different ways also. For instance, molecules that compose Pheromone $A$ are heavier than the ones that compose Pheromone $B$. In this case Pheromone $B$ is likely to diffuse through space faster than Pheromone A. How does this area of spread affect the colony's forage capability? The experiment in section \ref{sec:forage-radius-inv} investigates that. 

Pheromones are not the only means of ant communication, one of the other ways is by antennation. Ants have limited vision when compared with other animals, and their antennas serve as an effective way to 'see' each other. As will be discussed later on, different types of ants have different chemical coats, and their fellow ants are able to recognise these chemicals, therefore the type of their fellows. This is also used as a way to communicate. For example, perhaps a worker is following a pheromone trail that supposedly leads to a source of food, but suddenly it starts to come across many workers moving in the direction of the nest, and most of these workers are not carrying food. This could be interpreted as a sign of danger. If workers that are supposed to collect food, are running back to the nest without any, something ahead must be wrong. How does this second way of communication affect the velocity of reaction of the colony to danger? How does the variation of the number of workers going back to the nest without food as a threshold of danger situation affect the colony's reaction to danger? The experiment in section \ref{sec:warn-phero-inv} investigates that. 

In order to carry out these experiments, a flexible computational model that enables studies on social multi-agent complex systems is proposed. This model needs to be extensible and flexible enough to give users the freedom to create complex computational simulations. As in nature, the model formalises a decentralised set of entities, taking advantage of the tools computer models provide. The model also formalises an hierarchical infrastructure that can be used to describe different type of systems and track its components throughout simulations in order to acquire data.
 
In this model each agent is defined by its type and a list of tasks associated with them. The model is inspired in the subsumption architecture proposed by \citeauthor{1087032} \cite{1087032} \cite{Brooks1986b}. \citeauthor{1087032}' robots are composed by reactive interconnected modules, in a way that effectively gives different priorities to different modules. In the computational model proposed by this project, agents are composed by tasks, each task is completely independent from one another as far their execution goes. There is nothing from prevening a task to use another task to achieve a desired goal though. Differently from the subsumption architecture, the agents in the proposed model are not purely reactive. They are able to reason what they should do next and then select which task to execute if they are programmed to do so. The users have the freedom to implement the agents as they wish. Simple \ac{BDI} \cite{bratman1999intention} \cite{wooldridge2009introduction} agents can be implemented by extending the abstract infrastructure with minimum effort.

The environment in which these agents are placed is formed by a set of \emph{Nodes}. A node can be thought as a infinitesimal are of the environment. Nodes themselves are composed by a list of agents currently in the node and a list of communication stimuli, which represent any stimulus left by the agents that have been to the node. All simulations run for this paper used environments composed by $250,000$ interconnected nodes, but simulations using up to $750,000$ nodes have also been successfully run.

Agent-based modelling is at the core of the computational model. Even though more formal agent technology, such as communication protocols, is not used by the computational model, the very fact that agents are autonomous offers the ideal toolset to be used to create decentralised complex systems.

The model is presented in \ac{UML} as class diagrams and activity diagrams and it is implemented using the computer language Java. \ac{UML} has become the standard modelling language within the \ac{OOP} world as well as it has become a popular popular technique outside the \ac{OOP} circle \cite{fowler2004uml}. Among other features it provides a set of diagrams that can be used to describe classes, objects, action sequences and other parts of larger systems, such as packages. As far this project is concerned, class diagrams and activity diagrams supply all the tools needed to describe and document the model.

Java is a high level object-oriented programming language. It has been released publicly in 1995 and has developed to a powerful platform since then. Historically, Java has had great success in the enterprise environment, due to its flexibility, robustness and an entire ecosystem created around the core language that allows users to concentrate on the important parts of their work, leaving the language frameworks to deal with low level issues such as transaction management. In recent years, Java has been increasingly adopted by users from the academic background. With the introduction of libraries such as JScience and the development in hardware, it is possible to take advantage of features that are exclusive to Java to facilitate research in a variety of areas.

Most important for this project is the fact that Java offers a comprehensive, high level, shared-memory management and threading facilities. If not coding low level libraries users do not need to use complex synchronisation techniques such as semaphores. The language also offers key words, context blocks and utility libraries ready to be used when multi-threading is required.

In this project each agent is run as an isolated thread in any available CPU. Agents are completely autonomous from each other, and only share the same information about the environment around them, which means that, there is no access to global information, but only the local context is available to  decision making and task execution. The simulations executed for this project use from $10$ up to $300$ agents. There is no theoretical limit for the number of agents (that is threads) that could be used, only physical restrictions such as memory availability will impose a limit on the number of agents or nodes used in the simulations.