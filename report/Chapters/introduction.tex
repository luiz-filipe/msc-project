\chapter{Introduction}
\label{ch:introduction}

Recent studies have estimated that there are $8.7$ million eukaryote species on the planet.\cite{10.1371/journal.pbio.1001127} \cite{10.1371/journal.pbio.1001130} Most of them are still to be discovered. \citeauthor{dawkins1990selfish} argues that the utmost goal of every single one of them is survival, and to achieve that they do extraordinary things to solve the most varied range of challenges that the environment they live in imposes on them. Birds migrate thousands of miles in search of food, fish spend most of its the energy swimming up against the stream in order to reach the perfect location to lay their eggs. One could argue that these animals are intelligent. There are numerous ways to define intelligence though. \citeauthor{kennedy2001swarm} is part of the group of people who believe that intelligence is mostly about the capacity to adapt to new situations that have not been foreseen before. \cite{kennedy2001swarm} One shares this same point of view.

We are surrounded by examples of animals with a wide variety of complexity, from the very intellectually limited termites to us, humans beings, solving the problem of surviving in an ever changing environment in the most different and ingenious ways. It could be argued that humans are the most successful creatures wandering in this planet, it is very easy to blindly believe this statement, for the vast majority of us grow up being told that is the case. As beauty is in the eyes of the beholder, there are different ways to think about success though. If survival is all that matters, the number of individuals of certain species could be used as a good benchmark for success, and in this case humans are losing emphatically to insects. \cite{MAY16091988}

So how can these, as far as we are concerned, limited creatures accomplish very complex tasks? How can they react to the highly dynamic environments they live in? For example, when worker ants leave the nest to forage in the morning, the environment around the nest can be completely different from what it was last time they went out. Leafs might have fallen, thus changing how the nest can access food in the rain forest, the wind can blow the sand in the dessert, covering food sources and despite this unpredictable factors, they face the necessity to see off in search for food, and indeed they do. As a complex decentralised system, the nest is able to absolve the disruptions introduced by the environment and keep going on, doing its business. These questions are very hard to answer, and indeed this project has no intention to do so. 
 
Social insects are difficult to study for the very nature of the structures they create. These structures are decentralised and behaviours emerge from interactions between the system's agents. Accordingly, understanding of the agents' behaviour in isolation does not guarantee the understanding of the overall picture. Also, small changes in the way agents interact and react to external stimulus normally introduce great changes in the resulting overall behaviour; the complexity of these interactions and the amount of possible variables that can affect the system as a whole are so large that we easily fail to appreciate the incredible achievement that these complex systems are by themselves.

This project has as objective to study how agents and the ant colony as a whole react to changes in some properties of the pheromones that the ants use to communicate indirectly to each other. Generalising, ants use different pheromones to transmit different types of messages. As chemical substances, pheromones interact with the environment differently from one another, as a byproduct agents are affected in different ways also. For instance, molecules that compose Pheromone $A$ are heavier than the ones that compose Pheromone $B$. In this case Pheromone $B$ is likely to spread through space faster than Pheromone A. How does this spread area affects the colony's forage capability? The experiment in section \ref{sec:forage-radius-inv} investigates that. 

But pheromones are not the only means of ant communication, one of the other ways is by antennation. Naturally ants have limited vision when compared with other animals, and their antennas serve as an effective way to 'see' each other. As it will be discussed later on, different types of ants have different chemical coats, and their fellow ants are able to recognise these chemicals, therefore the type of their fellows. This is also used as a way to communicate. For example, if a worker is following a pheromone trail that supposedly leads to a source of food, but suddenly it starts to come across many workers moving in the direction of the nest, and most of these workers are not caring food. This could be interpreted as a sign of danger. If workers that are supposed to collect food, are running back to the nest without any, something ahead must be wrong. How does this second way of communication affects the velocity of reaction of the colony to danger? How does the variation of the number of workers going back to the nest without food as a threshold of danger situation affects the colony's reaction to danger? The experiment in section \ref{sec:warn-phero-inv} investigates that. 

In order to carry these experiments a flexible computational model that enables studies on social multi-agent complex systems is proposed. This model needs to be extensible and flexible enough to give users the freedom to create complex computational simulations. As in nature, the model formalises a decentralised set of entities, taking advantage of the tools computer models provide. The model also formalises an hierarchical infrastructure that can be used to describe different type of systems and track its components throughout simulations in order to acquire data.
 
In this model agents are defined by their type and a list of tasks associated to them. The model is inspired in the subsumption architecture proposed by \citeauthor{1087032}. \cite{1087032} \cite{Brooks1986b}. In the same way \citeauthor{1087032}'s robots are composed by reactive interconnected modules, in a way that effectively gives different priorities to different modules, in the computational model proposed by this project, agents are composed tasks, each task is completely independent from one another as far their execution goes. There is nothing to prevent a task to use another task to achieve a desired goal though. Differently from the subsumption architecture the agents from the proposed model are not purely reactive, they are able to reason what they should do next and then select which task to execute if they are programmed to do so. The users have the freedom to implement the agents as they wish. Simple \ac{BDI} \cite{bratman1999intention} \cite{wooldridge2009introduction} agents can be implemented by extending the abstract infrastructure with minimum effort.

Agent-based modelling is at the core of the computational model. Even though more formal agent technology, such as communication protocols, is not used by the computational model, the very fact agents are autonomous offers the ideal toolset to be used to create decentralised complex systems.

The model is presented in \ac{UML} as class diagrams and activity diagramsand it is implemented using the computer language Java. \ac{UML} has become the standard modelling language within the \ac{OOP} world as well as it has become a popular popular technique outside the \ac{OOP} circle. \cite{fowler2004uml}, among other features it provides a set of diagrams that can be used to describe classes, objects, action sequences and other parts of larger systems, such as packages. As far this project is concerned, class diagrams and activity diagrams supply all the tools it is needed to describe and document the model.

Java is a high level object-oriented programming language. It has been released publicly in 1995 and has developed to a powerful platform since then. Historically, Java has had great success in the enterprise environment, due to its flexibility, robustness and an entire ecosystem created around the core language that allows users to concentrate on the important parts of their work, leaving the language frameworks to deal with low level issues such as transaction management. In recent years, Java has been increasingly adopted by users from the academic background, with the introduction of libraries such as JScience and the development in hardware, it is possible to take advantage of features that are exclusive to Java to facilitate researches in a variety of areas.

Most important for this project is the fact that Java offers a very comprehensive, high level, shared-memory management and threading facilities. If not coding low level libraries users do not need to use complex synchronisation techniques such as semaphores. The language also offers key words, context blocks and utility libraries ready to be used when multi-threading is required.

In this project each agent is run as an isolated thread in any available CPU. Agents completely autonomous from each other, and only share the same information about the environment around them, which means that, there is not access to global information, but only the local context is available decision making and task task execution. The simulations executed for this project use from $10$ up to $300$ agents, there is no theoretical limit for the number of agents (that is threads) that could be used, only physical restrictions such as memory availability will impose a limit on the number of agents or nodes used in the simulations.