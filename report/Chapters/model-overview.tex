\chapter{Model Overview}
\label{ch:model-overview}

\section{Environment}

Explain how the environment is represented.

\subsection{Nodes}

Describe the Node interface and the BasicNode implementation. This implementation is a four-way connected grid.

\subsubsection{Pheromone Nodes}

Pheromone nodes are a specialisation of the BasicNode implementation that containing a list of chemical communication stimuli.

\subsection {Communication Stimuli}

Talk about CommunicationStimulus and CommunicationStimulusType interface. 
Explain why they are required (different types of communication can be designed and implemented). Relating to what is seen in nature.

\subsubsection{Chemical Communication Stimulus}

Define the new communication stimulus type and its properties as radius and decay factor. Relating to what is seen in nature.

\subsubsection {Forage Communication Stimulus}

Describe the type, its parameters and when it is used.

\subsubsection {Warning Communication Stimulus}

Describe the type, its parameters and when it is used.

\section {Agents}

A small introduction to agents in general, touch BDI agents, relate the feature of the agents of this model with the features of agents discussed in the introduction.

\subsection{Task Agents}

Explain that a specialisation of the Agent interface is proposed to create agents that are capable of executing tasks.

\subsection{Agent Types}

Explain what are agents types and why they are necessary. relate with cases in nature.

\section {Tasks}

Tasks are a unit of specialised work. Agents types have a list of tasks that a particular type is able to execute. Explain how agents can switch between tasks depending on the context.