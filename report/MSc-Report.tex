%%%%%%%%%%%%%%%%%%%%%%%%%%%%%%%%%%%%%%%%%
% Classicthesis Typographic Thesis
% LaTeX Template
% Version 1.0 (23/4/12)
%
% This template has been downloaded from:
% http://www.LaTeXTemplates.com
%
% Original author:
% AndrŽ Miede (http://www.miede.de)
%
% License:
% CC BY-NC-SA 3.0 (http://creativecommons.org/licenses/by-nc-sa/3.0/)
%
% General Tips:
% 1) Make sure to edit the classicthesis-config.file
% 2) New enumeration (A., B., C., etc in small caps): \begin{aenumerate} \end{aenumerate}
% 3) For margin notes: \marginpar or \graffito{}
% 4) Do not use bold fonts in this style, it is designed around them
% 5) Use tables as in the examples
% 6) See classicthesis-preamble.sty for useful commands
%
%%%%%%%%%%%%%%%%%%%%%%%%%%%%%%%%%%%%%%%%%

%----------------------------------------------------------------------
%	PACKAGES AND OTHER DOCUMENT CONFIGURATIONS
%----------------------------------------------------------------------

\documentclass[
		openright,titlepage,numbers=noenddot,headinclude,%1headlines,
                footinclude=true,cleardoublepage=empty,
                BCOR=5mm,paper=a4,fontsize=12pt, % Binding correction, paper type and font size
                ngerman,american, % Languages
                ]{scrreprt} 
                
% Includes the file which contains all the document configurations and packages - make sure to edit this file
\input{classicthesis-config}

%\usepackage[left=35mm,right=14mm,top=14mm,bottom=35mm]{geometry}

\begin{document}

\frenchspacing % Reduces space after periods to make text more compact

\raggedbottom % Makes all pages the height of the text on that page

\selectlanguage{american} % Select your default language - e.g. american or ngerman

%\renewcommand*{\bibname}{new name} % Uncomment to change the name of the bibliography
%\setbibpreamble{} % Uncomment to include a preamble to the bibliography - some text before the reference list starts

\pagenumbering{roman} % Roman page numbering prior to the start of the thesis content (i, ii, iii, etc)

\pagestyle{plain} % Suppress headers for the pre-content pages

%----------------------------------------------------------------------
%	PRE-CONTENT THESIS PAGES
%----------------------------------------------------------------------

\include{FrontBackmatter/Titlepage} % Main title page

%\include{FrontBackmatter/Titleback} % Back of the title page

\begin{addmargin}[+0.8cm]{-0.9cm}

%\cleardoublepage\include{FrontBackmatter/Dedication} % Dedication page

%\cleardoublepage\include{FrontBackmatter/Foreword} % Uncomment and create a Foreword.tex to include a foreword

\cleardoublepage% Abstract

\pdfbookmark[1]{Abstract}{Abstract} % Bookmark name visible in a PDF viewer

\begingroup
\let\clearpage\relax
\let\cleardoublepage\relax
\let\cleardoublepage\relax

\chapter*{Abstract} % Abstract name

The natural world offer examples in which very limited agents when put together can create very sophisticated overall behaviour. Social insects have been incredibly successful in solving complex problems. This project proposes an agent oriented computational model that simulates certain aspects of social insects living in a colony. A brief introduction is given on emergence and the relevant aspects of distributed complex systems to this project, as well as some background information on social insects and their communication techniques.

Two experiments are proposed and run. The first experiment investigates the relationship between agents and pheromone concentration present in the environment. The second experiment investigates the effect of changing the radius of action of chemical stimuli on the colony's foraging capability.

The resulting conclusions taken from the data collected point out the issues risen by the node selection process implemented. Another two experiments are proposed for further investigation on the node selection as well as possible improvements on the model related to the issues found during the simulations.

\endgroup

\vfill % Abstract page

%\cleardoublepage\include{FrontBackmatter/Publication} % Publications from the thesis page

%\cleardoublepage% Acknowledgements

\pdfbookmark[1]{Acknowledgements}{Acknowledgements} % Bookmark name visible in a PDF viewer

\begingroup

\let\clearpage\relax
\let\cleardoublepage\relax
\let\cleardoublepage\relax

\chapter*{Acknowledgements} % Acknowledgements section text

The challenge of taking a full-time postgraduate degree while working on a nearly full-time basis has proved to be very hard and I would have not be able to do it without the support and understanding of my manager Simon, who had a great deal of patience when I had to leave the office countless times to attend lectures and study for the exams.

I also would like to thank my close friends and family that made sure I was never short of support throughout the entire last year.

\endgroup % Acknowledgements page

\pagestyle{scrheadings} % Show chapter titles as headings

\cleardoublepage\include{FrontBackmatter/Contents} % Contents, list of figures/tables/listings and acronyms

\pagenumbering{arabic} % Arabic page numbering for thesis content (1, 2, 3, etc)
%\setcounter{page}{90} % Uncomment to manually start the page counter at an arbitrary value (for example if you wish to count the pre-content pages in the page count)

\cleardoublepage % Avoids problems with pdfbookmark

%----------------------------------------------------------------------
%	THESIS CONTENT - CHAPTERS
%----------------------------------------------------------------------

\ctparttext{In this first part, an introduction on the project is presented in Chapter \ref{ch:introduction}. Social insects and other background information,  such as emergence and agents are presented in Chapter \ref{ch:background-information}. Some of the applications of the concepts presented are highlighted at the end of the same chapter. The last chapter in this part, Chapter \ref{ch:model-overview}, introduces the generic computational model proposed by this paper and some parts of its implementation.} % Text on the Part 1 page describing  the content in Part 1


\part{Setting The Context} % First part of the thesis

\chapter{Introduction}
\label{ch:introduction}

Recent studies have estimated that there are $8.7$ million eukaryote species on the planet \cite{10.1371/journal.pbio.1001127} \cite{10.1371/journal.pbio.1001130}. Most of them are still to be discovered. \citeauthor{dawkins1990selfish} \cite{dawkins1990selfish} argues that the ultimate goal of every single one of them is survival, and to achieve that they do extraordinary things to solve the most varied range of challenges that the environment they live in imposes on them. Birds migrate thousands of miles in search of food, fish spend most of their  energy swimming up against the stream in order to reach the perfect location to lay their eggs. One could argue that these animals are intelligent, but here are numerous ways to define intelligence. \citeauthor{kennedy2001swarm} \cite{kennedy2001swarm} is part of the group of people who believe that intelligence is mostly about the capacity to adapt to new situations that have not been foreseen  \cite{kennedy2001swarm}, and one shares this same point of view.

We are surrounded by examples of animals with a wide variety of complexity, ranging from simple organisms with limited intelligence to us humans beings. All meet the challenged of surviving in an ever changing environment in  different and ingenious ways. It could be argued that humans are the most successful creatures on this planet. It is very easy to blindly accept this statement, for the vast majority of us grow up being told that is the case. Nevertheless, there are different ways to think about 'success'. If survival is all that matters, the number of individuals in a certain species could be used as a good benchmark for success, and in this case humans are losing emphatically to insects \cite{MAY16091988}.

So how can these, as far as we are concerned, 'limited' creatures accomplish very complex tasks? How can they react to the highly dynamic environments they live in? For example, when worker ants leave the nest to forage in the morning, the environment around the nest can be completely different from what it was last time they went out. Leaves might have fallen, thus changing the route to access food in the rain forest; the wind can blow the sand in the desert, covering food sources. Yet despite these unpredictable factors, the ants face the necessity to locate food, and indeed they do. As a complex decentralised system, the colony agents are able to overcome the disruptions introduced by the environment and keep going on, meeting the challenge of survival. These questions are very hard to answer, and indeed this project has no intention to do so. 
 
Systems of social insects are difficult to study due to the very nature of the structures they create. These structures are decentralised and behaviours emerge from interactions between the system's agents - the individual insects of which there are a great number. Accordingly, understanding of the agents' behaviour in isolation does not guarantee and understanding of the overall picture. Also, small changes in the way agents interact and react to external stimuli may introduce large changes in the resulting overall behaviour; the complexity of these interactions and the amount of possible variables that can affect the system as a whole are so large that we easily fail to appreciate the incredible achievement that these complex systems are by themselves.

This project has as objective to study how agents and the ant colony as a whole react to changes in some properties of the pheromones that the ants use to communicate indirectly to each other. In simple terms, ants use different pheromones to transmit different types of messages. As chemical substances, pheromones interact with the environment differently from one another. As a byproduct, agents are affected in different ways also. For instance, molecules that compose Pheromone $A$ are heavier than the ones that compose Pheromone $B$. In this case Pheromone $B$ is likely to diffuse through space faster than Pheromone A. How does this area of spread affect the colony's forage capability? The experiment in section \ref{sec:forage-radius-inv} investigates that. 

Pheromones are not the only means of ant communication, one of the other ways is by antennation. Ants have limited vision when compared with other animals, and their antennas serve as an effective way to 'see' each other. As will be discussed later on, different types of ants have different chemical coats, and their fellow ants are able to recognise these chemicals, therefore the type of their fellows. This is also used as a way to communicate. For example, perhaps a worker is following a pheromone trail that supposedly leads to a source of food, but suddenly it starts to come across many workers moving in the direction of the nest, and most of these workers are not carrying food. This could be interpreted as a sign of danger. If workers that are supposed to collect food, are running back to the nest without any, something ahead must be wrong. How does this second way of communication affect the velocity of reaction of the colony to danger? How does the variation of the number of workers going back to the nest without food as a threshold of danger situation affect the colony's reaction to danger? The experiment in section \ref{sec:warn-phero-inv} investigates that. 

In order to carry out these experiments, a flexible computational model that enables studies on social multi-agent complex systems is proposed. This model needs to be extensible and flexible enough to give users the freedom to create complex computational simulations. As in nature, the model formalises a decentralised set of entities, taking advantage of the tools computer models provide. The model also formalises an hierarchical infrastructure that can be used to describe different type of systems and track its components throughout simulations in order to acquire data.
 
In this model each agent is defined by its type and a list of tasks associated with them. The model is inspired in the subsumption architecture proposed by \citeauthor{1087032} \cite{1087032} \cite{Brooks1986b}. \citeauthor{1087032}' robots are composed by reactive interconnected modules, in a way that effectively gives different priorities to different modules. In the computational model proposed by this project, agents are composed by tasks, each task is completely independent from one another as far their execution goes. There is nothing from prevening a task to use another task to achieve a desired goal though. Differently from the subsumption architecture, the agents in the proposed model are not purely reactive. They are able to reason what they should do next and then select which task to execute if they are programmed to do so. The users have the freedom to implement the agents as they wish. Simple \ac{BDI} \cite{bratman1999intention} \cite{wooldridge2009introduction} agents can be implemented by extending the abstract infrastructure with minimum effort.

The environment in which these agents are placed is formed by a set of \emph{Nodes}. A node can be thought as a infinitesimal are of the environment. Nodes themselves are composed by a list of agents currently in the node and a list of communication stimuli, which represent any stimulus left by the agents that have been to the node. All simulations run for this paper used environments composed by $250,000$ interconnected nodes, but simulations using up to $750,000$ nodes have also been successfully run.

Agent-based modelling is at the core of the computational model. Even though more formal agent technology, such as communication protocols, is not used by the computational model, the very fact that agents are autonomous offers the ideal toolset to be used to create decentralised complex systems.

The model is presented in \ac{UML} as class diagrams and activity diagrams and it is implemented using the computer language Java. \ac{UML} has become the standard modelling language within the \ac{OOP} world as well as it has become a popular popular technique outside the \ac{OOP} circle \cite{fowler2004uml}. Among other features it provides a set of diagrams that can be used to describe classes, objects, action sequences and other parts of larger systems, such as packages. As far this project is concerned, class diagrams and activity diagrams supply all the tools needed to describe and document the model.

Java is a high level object-oriented programming language. It has been released publicly in 1995 and has developed to a powerful platform since then. Historically, Java has had great success in the enterprise environment, due to its flexibility, robustness and an entire ecosystem created around the core language that allows users to concentrate on the important parts of their work, leaving the language frameworks to deal with low level issues such as transaction management. In recent years, Java has been increasingly adopted by users from the academic background. With the introduction of libraries such as JScience and the development in hardware, it is possible to take advantage of features that are exclusive to Java to facilitate research in a variety of areas.

Most important for this project is the fact that Java offers a comprehensive, high level, shared-memory management and threading facilities. If not coding low level libraries users do not need to use complex synchronisation techniques such as semaphores. The language also offers key words, context blocks and utility libraries ready to be used when multi-threading is required.

In this project each agent is run as an isolated thread in any available CPU. Agents are completely autonomous from each other, and only share the same information about the environment around them, which means that, there is no access to global information, but only the local context is available to  decision making and task execution. The simulations executed for this project use from $10$ up to $300$ agents. There is no theoretical limit for the number of agents (that is threads) that could be used, only physical restrictions such as memory availability will impose a limit on the number of agents or nodes used in the simulations.
\chapter{Background Information}
\label{ch:background-information}

\section {Social Insects}

\section {Communication}
\label{sec:ant-comm}

tell that differnt pheromone types have different ranges and decay faster or slower.
warning must decay fast and have a greater action area than forage for example, explain why.

\subsection{Ants}



\section{Emergence}

Emergence is one of the concepts that are difficult to define. It is present in many different areas like science, arts and philosophy. It is seen throughout nature in phenomena such patterns on the sand in the desert and flock of birds.

A general concept of emergence can be defined as decentralised, local behaviour when seem from a higher perspective aggregates into a global behaviour. As local behaviour is not directly connected to the global behaviour, it does not matter to the aggregate outcome. That is, the agents that have local behaviour do not share a global behaviour as target.

There are multiple layers of emergence, and this is very important concept to understand complex systems.\cite{miller2007complex} As example one could take is any of most of the multi-cellular animals, including ourselves. The theoretical biologist \citeauthor{life1010034} argues that life is an emergent event itself. If we stop to think for a moment, there are trillions of cells in our body, each of them concerned only with their very specific context. Many of  them replaced daily, in a couple of years we are very likely to not have any cell that is in our body today, but we will continue to be what we are today, at least physically.

Emergence is not a new concept, it has been around for a long time. A good example of this is the Central Limit Theorem (CLT), which was first postulated in 1733 by the mathematician Abraham de Moivre. \cite{tijms2007understanding}. In few words the CLT states that, if certain conditions are satisfied, the mean of a large number of independent random variables will be distributed following a normal distribution.

\subsection{Types of emergence}

Typically emergence is split into weak and strong. It is possible to say that a phenomenon is strong emergent when it arises from a low-level domain, but the new qualities that these phenomenon bring to the system are irreducible to the system's constituent parts.\cite{laughlin2008different} As an example of strong emergence we can turn to ant colonies. Some ant colonies when are defending their queen, recruit workers to create a semi-sphere of ants around the queen, keeping it safe. The resulting global behaviour cannot be traced back to any individual worker. Now a weak emergence describes properties that can be reducible to its individual constituents. 

\subsection{Feedback}
\label{subsec:feedback}

When agents are interacting to each other, and these interactions are not independent, feedback becomes a very important part of complex systems. If the feedback is positive, disturbance on the system get amplified, leading to instability. A good example of positive feedback can be borrowed from Chemistry. In case of a chemical reaction happens faster at higher temperatures, but the reaction itself releases heat, it is very likely a positive feedback loop will be created and the reaction could lead to explosion very quickly.

On the other hand, if the feedback is negative, any disturbance on the system is absorbed, taking the system to a state of stability. There are many examples of positive feedback in our own body.Secretion of sweat to regulate body temperature and secretion of a variety of hormones to regulate from water absorption to salt absorption.  

\subsection{Decentralised Systems}

Systems that lack a central authority are called decentralised. In their most common form they are self-regulated, they are present in a vast range of domains, from nature to our society. Sock market is one example of such system. Although there are regulatory instruments in place to avoid abuse, the large number of dealers regulate the market as long as the value of shares are concerned. Imagine that for some reason a share is particularly attractive, so people are likely to buy it. Following the the high demand for the share, its price will rise. After some certain point, due its high price, the share will not be as much attractive anymore and agents involved in trading will go after other options. With time the demand for the share will get weaker and its price is likely to go down.

Of course this is an oversimplified version of what actually happens, but the important point here is that the agents involved in process of buying and selling the share regulate its price themselves. They are autonomous regarding taking the decision to buy or sell.

Arguably this property is the foundation of emergent systems. The autonomy of the system's components allows complex behaviour to emerge in a way that in centralised system it would not occur, actually in cases of centralised systems co-ordination is key. Indeed complex behaviour is capable of emerging in such systems, but only as a byproduct of this co-ordination. (((need and example here, but is going to be difficult to find)))

\section{Agent-based Object Models}

Agent-based modelling has proved to be one of the most relevant research areas in computing in the last decade. Nowadays we are overwhelmed by the amount of information available to us, and with the improvements on hardware in the last two decades gave some of the tools to use this available information in ways it was not possible before. Agent technology enters the scene taking advantage of these improvements and opens up a whole new world for new technologies to be created and put in use, wether for research or in  commercial applications. Different domains like biology, game theory, stock market and evolutionary computing are using agents extensively, from simulation on animal populations (((cite))) to predicting market patterns (((cite))).

An agent can be defined as a computational entity that is autonomous and exhibits flexible behaviour. Agents are also responsible over their own internal state. Usually agents are usually placed in environments that are dynamic and unpredictable. By flexible behaviour three main aspects are of most importance\cite{wooldridge2009introduction} :

\begin{enumerate}
\item Reactive: In most of the applications that agent technology is deployed the environment is not static. That is, it changes over time. A reactive system is capable of responding to the changes in the environment in the best way possible in order to the system to continue operating as it was before the changes had been introduced. 

\item Pro-active: Reacting only to a dynamic environment most of the times is not enough. Agents have have a reason to be, something to achieve, a goal. So it is crucial that agents not just react, but take the initiative to achieve their goals.

\item Social: Agents are likely to be deployed in a multi-agent environment, in some cases there can be some goals that are achieved only if agents cooperate with each other. Thus, social ability, that is being able to interact with other agents is vital.
\end{enumerate}

Agent technology provides a variety of standards and tools empowering designers and developers to structure applications around autonomous and communicative components from their concept to their implementation.\cite{al3roadmap}

All in all, agent-oriented modelling offer the best methodology and standards ecosystem for representing complex dynamic systems.

\subsection{Agents and Objects}

It is important to make the distinction between agents and objects. Objects are all about encapsulating state and providing methods to execute operation upon it. Objects do communicate, through messaging or method invocation, but they are passive. Objects and Object Oriented Programming (OOP) are merely the means to build agent systems. Agent-oriented modelling is a whole new programming paradigm. 

\subsection{Agents as a Theoretical Tool}

Another way to see Agent-oriented modelling is as a new kind of tool that empower us to touch questions that are very difficult to be addressed with traditional tools like mathematical methods. A tool that is particularly suited to deal with complex social systems.

In comparison with traditional tools, computational models sit at the other site of the spectrum. Traditional tools are static, precise and timeless; computational model are dynamic, flexible and timely. Even more, computational models are flexible enough in a way that we can add complexity to them in order to gain precision. So it is as if the computational model precision is a variable that we can control as we want or need to.

At a first moment it is hard to look at computational models as a science tool at the same level of mathematical methods that have been used for centuries to built good part of our knowledge of phenomena that surround us. But the problems we are committed to tackle today are different from the ones people were working on in the past and Agent-oriented modelling provides a powerful framework, which by its nature is well suited against this new class of complex problems.

\citeauthor{miller2007complex} present a in-depth discussion on the contrasts of traditional methods and computational models. As well as a list of the advantages of Agent-oriented modelling over traditional methods. 

\section{Parallel Programming}

\subsection{Threads And Task Execution}
\label{subsec:threads-task-exec}

In Java terms, threads are the mechanisms that are used to run tasks asynchronously, and it is a common mistake to think that the \emph{Thread} class is the primary abstraction for task execution in the language, but in fact the \emph{Executor} interface is. It is the base of a powerful task execution framework. It is important to note that the executors follow the producer-consumer pattern.

Java provides the \emph{Executors} factory to create thread pools for task execution. Using thread pools has many advantages over manually managing threads' lifecycle. It is possible to reuse threads to execute more than one task, what minimises the cost of creating and stoping threads, speeding up task execution. There are four methods provided by the \emph{Executors} factory for creating thread pools, for this project the most important are: \emph{newFixedThreadPool} and \emph{newScheduledThreadPool}. The former create a fixed-sized thread pool, tasks are executed as soon as submitted, when the fixed limit number is reached the tasks have wait until a thread is available to execute. The latter allows the creation of fixed-size thread pool as well, but in this case the pool supports delayed and periodic task execution.

The basic representation of a task in Java is the \emph{Runnable} interface, but tasks implementing \emph{Runnable} are not able to return a value or throw checked exceptions.\cite{goetz2006java} Now the \emph{Callable} interface is a richer abstraction of tasks, they allow the task to return a value and to throw checked exceptions. 

Tasks executed by an \emph{Executor} can have various states, as far as this project is concerned these states are not critical, because all the simulations run are to investigate the colony at a point in time only, so in all the cases, the thread pools will be created to the size of the number of agents necessary to run the simulation.

Now the \emph{Future} interface is an abstraction of the state of a task moving forward, it provides useful methods to manage and retrieve results of tasks.

So the natural way of executing tasks is to create a thread pool using an \emph{Executor} such as \emph{ExecutorService}, then submitting tasks to it, any classes that implement \emph{Runnable} or \emph{Callable}. The methods used for task submission are likely to return \emph{Future} objects that represent the task state. These objects can be used for a variety of things, like exception checking for instance.

\section{Current research}
\chapter{Model Overview}
\label{ch:model-overview}

\section{Environment}
\subsection{Nodes}
\subsection {Communication Stimuli}
\subsubsection{Chemical Communication Stimulus}
\section {Agents}
\subsection{Task Agents}
\subsection{Agent Types}

\section {Tasks}


\cleardoublepage % Empty page before the start of the next part

%------------------------------------------------

\ctparttext{In this part of the report two experiments are proposed and executed. Firstly an investigation on the colony's sensitivity in relation to present pheromone in the environment is carried. The objective of this experiment is to study how the agent's navigation capacity is affected by different concentrations of pheromone presented in the environment. 

The second experiment is a study on the affects that different diffusion rates of the forage pheromone will have on the colony's capability of foraging. The objective of this experiment is to determine if the colony's forage capacity is directly related to the increase of the pheromone radius. The results of each experiment are discussed in Chapter \ref{ch:experiments-and-observations}

Improvements to the model, to its implementation and further studies are proposed in Chapter \ref{ch:future-work}, followed by the conclusion of this paper.
}

\part{Experiments And Observations} % Second part of the thesis

\chapter{Experiments And Observations}
\label{ch:experiments-and-observations}

\section{Simulations and data collection}
\label{sec:running-sim}

The computational model does not formalise any resources to be used to create and run simulations, neither does the model formalise any way to collect data from the simulations.

JUnit is the reference unit test framework for Java, and the basic implementation of the proposed computational model is tested against more than 20 unit tests. However, unit tests also provide the perfect environment to run simulations. If resources such as initial node grid, initial pheromone concentrations and agents are going to be the same for more than one simulation, one can take advantage of test \emph{fixture} with the \emph{@Before} annotation to initialise these objects without repeating code. One could create a class for each simulation with a \emph{Main} method also, however it is more convenient to use the test framework.

Another critical aspect of the simulations is data collection. The model does not describe how data can be acquired during and after the simulation. For the simulations run for this paper logging was used for this purpose. Log4j is a thread-safe logging service maintained by the Apache foundation and it was used throughout the implementation and the simulations. In most of the simulations the logger was setup to write into files that are processed after the simulation is finished. Also, it is possible to configure the logger service to write on the console, so information on the agents can be easily visualised during the execution of the simulations.

\section{Pheromone Concentration Sensitivity}

Due to the model of node selection, agents are very sensitive to the pheromone concentration presented by the environment around them. This happens because the probability of choosing a neighbour node swings from $0\%$ to $100\%$ very swiftly, causing the agents to get trapped.

This experiment investigates how different initial pheromone concentrations in the environment and the amount of pheromone each agent is capable of depositing in each interaction affect the agents navigation through the space. It also has an objective to determine what values of these two parameters are acceptable to use in the other experiments.

Firstly, let's examine the case when all the nodes of the environment are created with no initial concentration of \emph{Forage} pheromone (Figure \ref{fig:initial-a}). When the agent is deciding which node it is going to make the first move to, all neighbours have the same probability of being picked, because of the $0$ of pheromone concentration (Figure \ref{fig:initial-b}). So as it was programmed to, the agent picks one node at random. But before moving to the next node it lays a bit of pheromone, suppose the concentration deposited is $0.1$. When the move is done, the environment around the agent should look like the one pictured in Figure \ref{fig:initial-c}.

\begin{figure}[H]
\myfloatalign
\subfloat[Initial environment]
{\label{fig:initial-a}
\includegraphics[width=.3\linewidth]{gfx/initial-01}} \quad
\subfloat[Initial probabilities]
{\label{fig:initial-b}
\includegraphics[width=.3\linewidth]{gfx/initial-02}} \\
\subfloat[Environment after move 1]
{\label{fig:initial-c}
\includegraphics[width=.3\linewidth]{gfx/initial-03}} \quad
\subfloat[Probabilities after move 1]
{\label{fig:initial-d}
\includegraphics[width=.3\linewidth]{gfx/initial-04}} \\
\subfloat[Environment after move 2]
{\label{fig:initial-e}
\includegraphics[width=.3\linewidth]{gfx/initial-05}} \quad
\subfloat[Probabilities after move 2]
{\label{fig:initial-f}
\includegraphics[width=.3\linewidth]{gfx/initial-06}}

\caption{Effect of initial pheromone concentration at zero}\label{fig:initial}
\end{figure}

Now it is time for the agent to move again. Firstly the agent reads the pheromone intensity in each of the neighbour nodes, after it computes where it is more likely to move. For all the nodes around the agent apart from the one it has just moved from have $0$ pheromone intensity in them, the agent is certain to move back to the previous node (Figure \ref{fig:initial-d}). Before completing the move, it lays pheromone in the current node. When the agent is back to the node it first started from, the scenario repeats and it becomes a cycle. The agent go back and forth, trapped in these two nodes forever, with no changes to explore the environment further. 

It is clear that the environment cannot be initialised with no \emph{Forage} pheromone in its nodes. The question that arises from this is: Is there any pair of values for these two parameters that will allow the agents to navigate in an acceptable fashion?

To answer this question, the experiment was setup with the following configuration:

\begin{table}[H]
\myfloatalign
\begin{tabularx}{\textwidth}{Xll} \toprule
\tableheadline{Property} & \tableheadline{Value} \\ \midrule
Number of lines & 500 \\
Number of columns & 500 \\
Total number of nodes &  250,000 \\
\midrule
Duration of each simulation & 10 s \\
Number of agents & 50 \\
Agent Type & WorkerType \\
Task executed & ForageTask \\
Agent sleep time & 5 ms \\
\bottomrule
\end{tabularx}
\caption{Experiment setup for investigation of initial pheromone concentration}  
\label{tab:setup-1}
\end{table}

In Table \ref{tab:setup-1} the property \emph{Agent sleep time} means how long the agent waits after a task is executed to choose another task to run. This is necessary to slow down agents (in this case the threads that are running the agent), otherwise they would cover the entire space in this 10 seconds only by the fact that they can do it very fast not because they are actively foraging.

The initial pheromone concentration and the amount of pheromone deposited in each interaction by the agents were varied from $0.001$ to $0.04$ in $5$ steps. Each of the possible pair of values for the two parameters has been simulated:

\begin{table}[H]
\myfloatalign
\begin{tabularx}{\textwidth}{XXXXX} \toprule
\tableheadline{1} & \tableheadline{2} & \tableheadline{3} & \tableheadline{4} & \tableheadline{5} \\ \midrule
0.001, 0.001 & 0.001, 0.005 & 0.001, 0.01 & 0.001, 0.02 & 0.001, 0.04 \\
0.005, 0.001 & 0.005, 0.005 & 0.005, 0.01 & 0.005, 0.02 & 0.005, 0.04 \\
0.01, 0.001 & 0.01, 0.005 & 0.01, 0.01 & 0.01, 0.02 & 0.01, 0.04 \\
0.02, 0.001 & 0.02, 0.005 & 0.02, 0.01 & 0.02, 0.02 & 0.02, 0.04 \\
0.04, 0.001 & 0.04, 0.005 & 0.04, 0.01 & 0.04, 0.02 & 0.04, 0.04 \\
\bottomrule
\end{tabularx}
\caption{Variations for initial concentration and amount of pheromone deposited by agents}  
\label{tab:setup-2}
\end{table}

In each case, the colony containing $50$ agents is created at the north of the environment, horizontally centred. The agents execute the \emph{ForageTask} since their creation, they do not analyse any contextual parameters such as other agents, they only try to move through the space using the rules defined by the task.

In order to compare how each possible value pairs in Table \ref{tab:setup-2} affect the resulting navigation of the agents, two samples of  the pheromone trail left by the agents are analysed. The first one is from close to the nest that will enable us to check how is the agents' response to the initial pheromone concentration shortly they have left the nest. The second sample is taken further ahead in the environment, far from the nest. This sample is a good way to test how the agents' own deposit of pheromone will affect the system behaviour, Figure \ref{fig:initial-sample} illustrates how and where the samples are made.

\begin{figure}[H]
  \centering
  \includegraphics[width=0.4\linewidth]{gfx/initial-sample.png}
  \caption{How the two samples of the environment is made}
  \label{fig:initial-sample}
\end{figure}

Figures \ref{fig:initial-var-close} and \ref{fig:initial-var-far} show the resulting sampling for all possible combination for the initial pheromone concentration and the amount of pheromone deposited by each agent in each interaction.

\begin{figure}[H]
\myfloatalign
\subfloat[]
{\label{fig:initial-var-close}
\includegraphics[width=.6\linewidth]{gfx/initial-variations-close}} \\
\subfloat[]
{\label{fig:initial-var-far}
\includegraphics[width=.6\linewidth]{gfx/initial-variations-far}} \\

\caption{Samples of the pheromone trail left by the agents close to the nest (a) and far from the nest (b). }\label{fig:initial-var-final}
\end{figure}


The trails left by the colonies can be compared in relation on how strong they are, how the agents are able to 'scape' them to explore the environment and how it shaped when agents get further from the nest.

Starting from $0.001$ as the amount of pheromone to be deposited by agents in each interaction; it is possible to observe from the figures that two very contrasting behaviours emerge, firstly because the environment has so little pheromone and the update is so small, they weight assigned to each of the neighbour nodes count considerably more than the pheromone deposited by the agents, so the agents end up very dispersed, thus no chemical trail is formed at all. This phenomena is actually seen in many other combination of the parameters, all the cases when the update is $0.001$ in fact. 

When the amount of pheromone deposited by the agents is increased the behaviour of the colony could not be more different than what was seen previously. The agents switch from exploring a large area to be 'trapped' into the pheromone trail. This impedes the agents of exploring the space, what is not desirable for any colony. This behaviour is also seen in other values for the parameters. What seems to be the rule is that if the update is considerable larger than the concentration of pheromone in the environment the agents will start to create a 'bubble' of high pheromone concentration and as consequence they are very unlikely to move to any node outside this area.

It rises the question, why does it happen? The answer is similar to one of the problem in initialising the environment with no pheromone at all. In this case, the critical point is the rate between the initial concentration and the amount deposited by the agents in each interaction.

Figure \ref{fig:update} illustrates how swiftly the probabilities can change depending on the amount of pheromone deposited in the node by agents. The node in red in the picture represents a node that has been updated previously by another agent. In the first scenario (Figure \ref{fig:update-c}) the update was $0.001$, in the second (Figure \ref{fig:update-e}) the pheromone deposited was $0.01$. The probability of the node in red being picked up to be the next node the agent will move to more than doubled. (Figure \ref{fig:update-d} and Figure \ref{fig:update-f}).

\begin{figure}[H]
\myfloatalign
\subfloat[]
{\label{fig:update-a}
\includegraphics[width=.25\linewidth]{gfx/update-01}} \quad
\subfloat[]
{\label{fig:update-b}
\includegraphics[width=.25\linewidth]{gfx/update-02}} \\
\subfloat[]
{\label{fig:update-c}
\includegraphics[width=.25\linewidth]{gfx/update-03}} \quad
\subfloat[]
{\label{fig:update-d}
\includegraphics[width=.25\linewidth]{gfx/update-04}} \\
\subfloat[]
{\label{fig:update-e}
\includegraphics[width=.25\linewidth]{gfx/update-05}} \quad
\subfloat[]
{\label{fig:update-f}
\includegraphics[width=.25\linewidth]{gfx/update-06}}

\caption{Shift of probability depending on agent update}\label{fig:update}
\end{figure}

Further investigation revealed that the probability of selection increases in a logarithmic fashion. Figure \ref{fig:prob-inc} shows how the increase in probability progress when the amount of pheromone deposited by the agents increases by multiples of the initial concentration in the environment.

\begin{figure}[H]
  \centering
  \includegraphics[width=0.6\linewidth]{gfx/probability-increase.png}
  \caption[Selection probability increasing]{Increase of probability selection according to the increase of the deposit increment}
  \label{fig:prob-inc}
\end{figure}

It is very tempting to conclude that if the right ratio between the initial pheromone concentration and the increment step used by the agents is found the, problems of the lack of trail formation and the agent confinement in the trail will be solved. However from figures \ref{fig:initial-var-close} and \ref{fig:initial-var-far} it is observed that the same ratio (initial concentration/increase step) present different results for different values. In the case $0.001/0.001$ no trail is formed at all, for $0.01/0.01$ a solid trail is formed. 

The experimental results are that the initial pheromone concentration should not be too low that avoid trail formation or saturation. It has to be an intermediate value that allows the agents to converge to a specific path, generating the trail, but not too fast, allowing the agents to explore other parts of the environment as well. As far as the update step goes, it also needs to be an intermediate value, so it starts actually being part of the node selection process but not big enough to quickly saturate the pheromone trail.

\begin{figure}[H]
\myfloatalign
\subfloat[]
{\label{fig:trail-a}
\includegraphics[width=.1\linewidth]{gfx/trail-001}} \quad
\subfloat[]
{\label{fig:trail-b}
\includegraphics[width=.1\linewidth]{gfx/trail-01}} \quad
\subfloat[]
{\label{fig:trail-c}
\includegraphics[width=.1\linewidth]{gfx/trail-0201}} \quad
\subfloat[]
{\label{fig:trail-d}
\includegraphics[width=.1\linewidth]{gfx/trail-0202}} \quad
\subfloat[]
{\label{fig:trail-e}
\includegraphics[width=.1\linewidth]{gfx/trail-0404}} \quad

\caption[Example of complete pheromone trails]{Complete trail pheromone trails left by simulation using 0.001,0.04 (a), 0.01,0.01 (b), 0.02,0.01 (c), 0.02,0.02 (d) and 0.4,04 (e) for initial pheromone concentration and update step respectively}
\label{fig:trails}
\end{figure}

One can conclude from this analysis pair $0.01/0.01$ (Figure \ref{fig:trail-b}) for the initial pheromone concentration and increment step present the best results. It allows pheromone trail to \emph{emerge as a result of the agents' interactions with the environment}. It could also be argued that another pairs such $0.01/0.02$ and $0.02/0.02$ set the right conditions for agent navigation. However, in this experiment the agents are executing the \emph{ForageTask} without actually collecting food, to no agent tries to deposit food in the nest and start foraging again - the trail is not reinforced. Taking that in consideration the other pairs tested could also be considered to lead to a premature saturation of the trail, which is the case of $0.02/0.02$.

The pair $0.01/0.01$ also proved to be less sensitive to the number of agents used in this and the other experiments.

\section{Forage Radius Investigation}
\label{sec:forage-radius-inv}

This experiment investigates the affect of the radius variation of the \emph{Forage} chemical stimulus (section \ref{subsec:forage-stimulus}) on the capacity of the colony to collect food. Four values for the stimulus' radius were tried - $0$, $1$, $2$. The experiment was setup as follows:

\begin{table}[H]
\myfloatalign
\begin{tabularx}{\textwidth}{Xll} \toprule
\tableheadline{Property} & \tableheadline{Value} \\ \midrule
Number of lines & 500 \\
Number of columns & 500 \\
Total number of nodes &  250,000 \\
Initial grid pheromone intensity & 0.01 \\
\midrule
Pheromone radius variations & 0, 1, 2 \\
Duration of each simulation & 30 s \\
Number of agents & 50 \\
Agent Type & WorkerType \\
Pheromone increment step & 0.01 \\
Task executed & ForageTask and FindHome\\
Agent sleep time & 5 ms \\
Pheromone decay frequency & every 3 seconds \\
\bottomrule
\end{tabularx}
\caption{Experiment setup for pheromone radius variation study}  
\label{tab:setup-2}
\end{table}

In this experiment the radius of action of the forage pheromone is varied in order to check the effects in the amount of food the colony is capable of forage. At first one would expect that the increase on the radius of the forage pheromone would have a positive impact on the colony capacity of collecting food as trails that lead to food sources would are reinforced by agents caring food back to the nest and with a wider spread of the pheromone more workers would be fall in these trails. However the data from the simulations show a different picture. The colony collect less and less food as the stimulus' radius increase. Table \ref{tab:radius-results} is a summary of the data collected from the study.

For this experiment a row of food sources was placed far away from the colony. Each source has a total amount of food of $30$. The simulation was run $100$ times for each radius variation. Figure \ref{fig:radius-res} presents part of the middle section (see Figure \ref{fig:radius-full-a}) of the trails created by the agents for the first $20$ simulations for each value used for radius.

\begin{figure}[H]
\myfloatalign
\subfloat[radius = 0]
{\label{fig:radius-res-a}
\includegraphics[width=.8\linewidth]{gfx/radius-res-0}} \\
\subfloat[radius = 1]
{\label{fig:radius-res-b}
\includegraphics[width=.8\linewidth]{gfx/radius-res-1}} \\
\subfloat[radius = 2]
{\label{fig:radius-res-c}
\includegraphics[width=.8\linewidth]{gfx/radius-res-2}} \\

\caption[Visual samples of variation of radius]{20 graphical samples of the results in variating forage stimulus' radius}\label{fig:radius-res}
\end{figure}

\begin{figure}[H]
\myfloatalign
\subfloat[setup]
{\label{fig:radius-full-a}
\includegraphics[width=.15\linewidth]{gfx/radius-experiment}} \quad
\subfloat[radius = 0]
{\label{fig:radius-full-b}
\includegraphics[width=.15\linewidth]{gfx/radius-full-0}} \quad
\subfloat[radius = 1]
{\label{fig:radius-full-c}
\includegraphics[width=.15\linewidth]{gfx/radius-full-1}} \quad
\subfloat[radius = 2]
{\label{fig:radius-full-d}
\includegraphics[width=.15\linewidth]{gfx/radius-full-2}}

\caption{Example of trail formation for different radiuses}
\label{fig:radius-full}
\end{figure}

\begin{table}[H]
\myfloatalign
\begin{tabularx}{\textwidth}{Xll} \toprule
\tableheadline{Radius} & \tableheadline{Average of food collected} & \tableheadline{Standard deviation} \\ \midrule
0 & 6.0 & 1.21 \\
1 & 4.9 & 0.72 \\
2 & 3.3 & 0.53 \\

\bottomrule
\end{tabularx}
\caption{Forage radius simulations' outcomes}  
\label{tab:radius-results}
\end{table}

The averages that are presented in the Table \ref{tab:radius-results} was calculated after statistically validating the dataset collected for each radius value, with any outlier sample was removed from the results collected from the simulations. The following pair of equation are used to define a sample point $x_i$ as outlier or not:

\begin{equation}
\begin{cases} 

x_{i} \leq \bar{x} - 2 \times \sigma \\
x_{i} \geq \bar{x} + 2 \times \sigma

\end{cases}
\end{equation}

Where $\bar{x}$ is the average of the resulting dataset for each radius and $\sigma$ is the standard deviation. The data collected proved to be consistent and only few samples were rejected. Figure \ref{fig:radius} shows the samples spread for each radius and their probability density distribution.

\begin{figure}[H]
\myfloatalign
\subfloat[]
{\label{fig:radius-a}
\includegraphics[width=.4\linewidth]{gfx/radius-0}} \quad
\subfloat[]
{\label{fig:radius-b}
\includegraphics[width=.4\linewidth]{gfx/radius-1}} \\
\subfloat[]
{\label{fig:radius-c}
\includegraphics[width=.4\linewidth]{gfx/radius-2}} \quad
\subfloat[]
{\label{fig:radius-d}
\includegraphics[width=.4\linewidth]{gfx/radius-pdf}}

\caption[Radius variation - samples distributions]{Samples distribution variating the pheromone radius - 0 (a), 1 (b) and 2 (c) - and the samples' probability density distribution (d)}\label{fig:radius}
\end{figure}

Figure \ref{fig:radius} helps to visualise what the standard deviations from Table \ref{tab:radius-results} had already shown - the samples \emph{variance} decreases when the radius increases. The reason for this phenomena is that the bigger the pheromone radius is the lesser agents are probable to 'escape' from the trail, thus they are not able to explore different areas of the environment. For the agents use virtually only the same pheromone trail to forage their outcomes are likely to be very close.  

With a more diversified exploratory reach, agents in the simulations using radius $0$ are likely to have different degrees of success in finding food and taking it back to the nest in each simulation run, justifying the greater variance in the colony outcome.

The study carries raises another question - why does smaller radiuses outperform bigger ones? From Table \ref{tab:radius-results} we can see that the simulations with radius $0$ outperforms the ones with radiuses $1$ and $2$ in $22$ and $82$ percent in average respectively. This occurs due to the formation of trails with large width, leading the agents to move from note to node in almost random fashion. 

Figure \ref{fig:radius-res} illustrates how the pheromone trails widen with the increase of the pheromone radius. It might seem contradictory at first that wider trails do not led to better forage performance, for more ants should be lead to food sources. However due to the model of node selection implemented the agents are too sensitive to the pheromone intensity of the direct neighbours of the node they are currently at, and this is translated to random node selection when agents are trapped in wide pheromone trails. 

\begin{figure}[H]
\myfloatalign
\subfloat[]
{\label{fig:radius-sat-a}
\includegraphics[width=.3\linewidth]{gfx/radius-sat}} \quad
\subfloat[]
{\label{fig:radius-sat-b}
\includegraphics[width=.3\linewidth]{gfx/radius-sat-nav}}

\caption{Affect of wide pheromone trails on node selection}\label{fig:radius-saturation}
\end{figure}

This random behaviour, as far node selection is concerned, has a profound impact on the colony's forage performance, as agents will struggle to find their way to the food sources, and when eventually some of them do, there still a good change that many will not reach the nest back in order to deposit the food they have collected.  

This strongly suggests that ants use more than only local information available at the instant they are making decision where to move to. The other communication methods employed by ants should play a vital role in assisting this decision to be made.

Figure \ref{fig:radius-explored} is a renderisation  of the space visited by all ants of the colony throughout the simulation (left frame) and the space visited by only one of the agents of that colony (right frame), set to be tracked by the framework. As it is possible to see, as the radius increases so does the number of paths the agents visit and more tortuous they become. Figures \ref{fig:radius-exp-b} and \ref{fig:radius-exp-c} also show how the ants get 'trapped' in the pheromone cloud. In both figures, the paths formed by the single agent (right frame) have the same shape of the main stream of paths visited by the entire colony (left frame). 


Two new possible experiments to better understand and improve node selection are proposed in section \ref{sec:new-studies}.

\begin{figure}[H]
\myfloatalign
\subfloat[]
{\label{fig:radius-exp-a}
\includegraphics[width=.25\linewidth]{gfx/radius-explored-0}} \quad
\subfloat[]
{\label{fig:radius-exp-b}
\includegraphics[width=.25\linewidth]{gfx/radius-explored-1}} \quad
\subfloat[]
{\label{fig:radius-exp-c}
\includegraphics[width=.25\linewidth]{gfx/radius-explored-2}}

\caption{Examples of the space explored by the entire colony (left) and by one agent only (right). Varying the stimulus' radius form $0$ (a) and $1$ (b) to $2$ (c).}\label{fig:radius-explored}
\end{figure}

%\section{Warning Pheromone Response}
%\label{sec:warn-phero-inv}
%In this experiment the response of the agents to a warning communication stimulus is tested.

%Describe the experiment setup.

%Explain that the experiment is done in two phases, the first one the agents react only to amount of warning pheromone and the second one the agents react also to the number of other agents they meet that are traveling in the opposite direction and are not caring food.

%Add block diagrams that explain the algorithms the agents use to decide on task selection.

%Discuss the results. [experiment not done yet]
%Results should vary with the use of different parameters, such as the number of agents traveling in the opposite direction before the agent abort the current task and change to findNestAndHide task.

\chapter{Future Work and Conclusion}
\label{ch:future-work}

\section{Model Improvements}

The model proved to be flexible enough to enable easy implementation of the base classes and the declaration of different agent types and resources. The area that has been least developed is related with simulation. As explained in section \ref{sec:running-sim} there is no formalisation related to simulations. This did not prove to be of any inconvenience during the creation and execution of the simulations for this paper. 

However, for users who are less acquainted with the model and its implementation it could prove to be a challenge to write simulations and collected data from them. More important is the fact that managing all the elements that compose a simulation would be very challenging for the users less experienced to the Java language itself.

\subsection{Simulation handler}

Formalising the simulations and data acquisition by the use of interfaces is necessary. The would remove from the user the responsibility of creation and management of all elements necessary to run simulations. This could be done by the creation of a \emph{simulation handler} that would contain all the necessary data objects, such as the environment (grid of nodes), list of agents, simulation renderers and the necessary methods to schedule data collection, execution of chemical stimulus renderers and so forth.

\subsection{Ant agent navigation improvements}

When it came to implement the different agents from the different casts, the biggest challenge faced was to create an algorithm to take the agents back to their nest. The method implemented in this paper (see sections \ref{sec:ant-memory}, \ref{task:find-home-hide} and \ref{task:find-home-hide}) should be improved in order to be used in more complex and longer simulations. In some cases the simulations would have a high rate of ants not finding their nest at all.

A possible improvement is to bring chemical landmarks to the model. An agent type could be created for this. This type could have properties that would point the ants to the right direction to the nest.

\subsection{Nests As Agents}

As seen in section \ref{sec:ant-nest} nest are represented as agents. On the one hand this facilitates their declaration and use as they enjoy all the infrastructure already in place for agents. On the other hand they are punctual, that is, they are placed in a node as any other agent and the other agents can see their nests only if they reach the same node at which the nest is placed. This is clearly a disadvantage and it does not reflect the reality of natural nests also.

An improvement to the model, that could be used to solve the problem of nests being punctual, could be an new interface to declare \emph{node types}. A node would have a type in the same way agents have. This would open up opportunity to create far more complex environments that it is possible now. For instance, a node could be an  \emph{obstacle} type, agents would not be allowed in there, or even a node could be a chemical landmark. With nodes having types, nests could be modelled as a set of nodes of the \emph{nest type}. This nodes would form a rectangular shape grid, in which if an agent reach any of the nodes they would be able to recognise they have reached their nest.

\section{Implementation Issues}

There is room from improvement on the basic implementation of the model. The comments made in the source files should be consolidated and in some cases improved. However a major possible improvement on the basic implementation is to change the ways the nodes are connected to from the environment grid.  

Currently the nodes are connected in a \emph{four-way} fashion, each node has $4$ neighbours, one in each direction: \emph{north, east, south, west}. This creates a major limitation related to agent navigation. Agents are very unlikely to move in diagonals, as it consists in two movements, first the agent move forward and after it moves sideways. It has been observed that agents hardly would do that because they have very low probability of selecting nodes in a diagonal sequence, e.g. north, east, north, east, north, east. Thus agents generally could not find any food sources in more complex environments where they were not directly aligned to the nest where the ants departed from.

\begin{figure}[H]
\myfloatalign
\subfloat[]
{\label{fig:connection-a}
\includegraphics[width=.2\linewidth]{gfx/connection-4}} \quad
\subfloat[]
{\label{fig:connection-b}
\includegraphics[width=.2\linewidth]{gfx/connection-8}}

\caption{Node connection to its neighbours, four-way connected grid (a) and the eight-way connected grid (b)}
\label{fig:connection}
\end{figure}

An implementation of a grid connected differently would improve the problem considerably. If nodes had $8$ neighbours instead of $4$, agents would be able to navigate directly in the diagonal. Figure \ref{fig:connection} illustrates how the nodes would be connected and all the possible directions the agents could travel towards.

\section{Proposed Studies}
\label{sec:new-studies}

\subsection{Limited Number of Agents Per Node}

This experiment has the potential to explore how the formation and saturation of the pheromone trails would be affected by the limiting the number of agents that can be in a same node at the same time. From the experiment of section \ref{sec:forage-radius-inv} it is clear that in the case of a chemical communication stimulus having radius greater than $0$ agents do not follow a \emph{rational} behaviour towards their goal of foraging. As explained in the experiment discussion the node selection model is responsible for that.

It would be interesting to investigate how the trails would emerge if the nodes had a maximum number of agents allowed in that node at the same time, this would force some agents to move to nodes they would not move to otherwise, it is clear that this is the actual case in nature, as two ants cannot occupy the same physical space. 

\subsection{Improved pheromone sensing capability}

The experiment \emph{Forage Radius Investigation} (ection \ref{sec:forage-radius-inv}) has showed that the agents are too sensitive to the pheromone intensity of the direct neighbours of the node they are currently at. Again, this is due to the node selection model.

This proposed experiment would investigate how the colony forage performance and trail formation would change if agents would take in consideration not only the pheromone intensity of the neighbours of its current node but also the neighbours of the neighbours would be part of the decision process. Different 'depths' of sensibility could be experimented. This new feature would reduce the agent's sensibility to direct neighbours of its current node. 

Another possible improvement on the agent's sensing capacity would be the introduction of some sort of memory that would play part of the node selection process.

\section{Conclusion}

The proposed computational model proposed by this paper proved to achieve the two objectives set for this project - flexibility and robustness. The generic model infrastructure provided all the necessary entities to define and implement a wide variety of agents, environments and tasks. More than that, it was possible to demonstrate how simple it is to extend the model, using the case of ant colonies and 2 dimensional environments. Layers of complexity were added as needed, taking advantage of all the previously existing infrastructure with minimal effort.

Also, simulations that were run repeatedly using the same parameters returned, within the expected variations, the same results. The simulations scaled well when the number of agents and the size of the environment varied greatly. Some test experiments used as few as 5 agents in small environments containing a few hundreds of nodes. Some simulations were run for minutes using up to 350 agents in environments containing up to 750,000 nodes with no problems at all.

As far as the implementation of the model is concerned, the model of node selection implemented by this paper presented to be inappropriate. Further study should be carried (see section \ref{sec:new-studies} for proposed studies) in order to confirm that. The agents' over-sensitivity to the neighbours of the node it is currently in, is the most problematic effect of the node selection model. The necessity of having to initialise the environment with some pheromone intensity, for the smallest that it might be, is a strong argument against the node selection model implemented by itself. It does not reflect what actually happens in real ant colonies. However this did not affected the two proposed experiments in anyway as their objective was to study how the agent's react to the environment changes, starting from a pre-defined setup, whatever it might be.

As for the agent implementation, the problem of taking the agents back to their nest proved to be by far the most challenging task to be implemented. The tasks \emph{FindHomeTask} (section \ref{task:find-home}) and \emph{FindHomeAndHide} (section \ref{task:find-home-hide}) use the agent's limited memory to direct the  agent on its way to the nest, however they proved to be very inefficient and other mechanisms such as chemical landmarks must be considered.

The first experiment demonstrated how agents react to different concentrations of pheromone, the most important outcome of the experiment was the contestation that the trails are a by product of the agents' interactions not only with the environment but also with other agents. They clearly \emph{emerge} from these interactions. This is a case of \emph{strong} emergence. 

Another noteworthy conclusion taken from the first experiment is that the relationship between the quantity of pheromone that each agent deposit in each interaction and the amount of pheromone already present in the environment regulate the colony's capability to explore the environment around its nest. There is a fine line between having very successful colonies, as far space exploration is concerned, and colonies that could not even reach a source of food. The colony's behaviour switches from one state to another very swiftly depending on this rate.

The second experiment showed that the increase of the communication stimulus' radius did not mean an increase on the forage capacity of the ant colonies, as it was firstly expected. Smaller radiuses give rise to thin, better defined, pheromone trails that allow the agents to move towards the food source direction objectively. Whereas, larger radius 'confuse' the agents within the pheromone trail, seriously undermining the colony's capacity of foraging. The reason for this problem is that communication stimuli with larger radius of reach end up creating clouds of pheromone, and agents have little change to escape this region of high pheromone concentration. This creates a vicious cycle, for the agents in the trail continue to deposit pheromone, over and over again around the same area, saturating the nodes within the trail with pheromone very quickly. Resulting in even smaller probability of 'breaking' from the trail. It is clear that this outcome is also closely related to the agents' over-sensitivity to the pheromone intensity in the neighbour nodes of the current node the agent is in. Another strong result against the model of node selection implemented. % Chapter 3
%\include{Chapters/Chapter04} % Chapter 4 - empty template

%----------------------------------------------------------------------
%	THESIS CONTENT - APPENDICES
%----------------------------------------------------------------------

\appendix

\part{Appendix} % New part of the thesis for the appendix

\chapter{Extra Experimental Results}

I need to select which images to add here. They already have been generated, I just need to choose and crop some of them if necessary.

\section{Forage radius investigation}
\label{ap:exp-2}

\begin{figure}[H]
  \centering
  \includegraphics[width=0.8\linewidth]{gfx/radius-0-final.png}
  \caption{Full forage trails with radius at 0}
  \label{fig:radius-0-final}
\end{figure}

\begin{figure}[H]
  \centering
  \includegraphics[width=0.8\linewidth]{gfx/radius-1-final.png}
  \caption{Full forage trails with radius at 1}
  \label{fig:radius-1-final}
\end{figure}

\begin{figure}[H]
  \centering
  \includegraphics[width=0.8\linewidth]{gfx/radius-2-final.png}
  \caption{Full forage trails with radius at 2}
  \label{fig:radius-2-final}
\end{figure}
\singlespacing
\chapter{Model and Simulation Source Code}

\section{Model Implementation Details}
\section{Source Code}
\doublespacing

%----------------------------------------------------------------------
%	POST-CONTENT THESIS PAGES
%----------------------------------------------------------------------

\cleardoublepage\include{FrontBackmatter/Bibliography} % Bibliography

%----------------------------------------------------------------------
\end{addmargin}
\end{document}
