\chapter{Introduction}
\label{ch:introduction}

Recent studies have estimated that there are 8.7 million eukaryote species on the planet.\cite{10.1371/journal.pbio.1001127} \cite{10.1371/journal.pbio.1001130} Most of them still to be discovered. \citeauthor{dawkins1990selfish} argues that the utmost goal of every single of them is survival, and to achieve that individuals do extraordinary things to solve the most varied range of challenges that the environment they live impose to them. Birds will migrate thousands of miles in search plenty of food, fishes spent almost all the energy they have to swim agains the stream in order to reach the perfect location to lay their eggs. \citeauthor{kennedy2001swarm} is part of the group of people that believe that intelligence is mostly about the capacity to adapt to new situations that were not foreseen before. \cite{kennedy2001swarm} One shares the same thought.

We are surrounded by examples of animals with a wide variety of complexity, from the very intellectually limited termites to us humans beings, solving the problem of surviving in a ever changing environment in the most different and ingenious ways. It could be argued that humans are the most successful creatures wandering the planet, it is very easy to blindly believe this statement, for the vast majority of us grow up being told that. As beauty is in the eyes of the beholder, there are different ways to think about success though. If survival is all that matters, so the number of individuals of a certain species could be used as a good the benchmark for success, and in this case humans are loosing emphatically to insects. (((need reference)))

So how can these, as far as we are concerned, limited creatures accomplish very complex tasks? How can they react to the highly dynamic environment they live on? For example, when worker ants leave the nest to forage in the morning, the environment around the nest could be completely different from what it was last time they went out. Leafs might have fallen and changed how the nest can access food in the rain forest, the wind blows the sand in the dessert, covering food sources and so on, any way, they go out next time and they must found food, and indeed they do. As a complex decentralised system, the nest is able to absolve the disruptions introduced by the environment and keep going on doing its business. These questions are very hard to answer, indeed this project has no intention to do so. 
 
Social insects are difficult to study for the very nature of the structure they create. They are decentralised and behaviours emerges from the interaction between the system's agents. What means that understanding the agents' behaviour in isolation does not guarantee the understanding of the overall picture. Also, small changes in the way agents interact and react to external stimulus normally introduce great changes in the resulting overall behaviour, the complexity of these interactions and the amount of possible variables that can affect the system is so large that we easily fail to appreciate the incredible achievement that these complex systems by themselves are.

This project has as objective to study how agents and the ant colony as a whole react to changes in some properties of the pheromones that the ants use to communicate indirectly to each other. Generalising, ants use different pheromones to transmit different types of messages. As chemical substances, pheromones interact with the environment differently from one another, as a byproduct agents are affected in different ways also. Imagine that the molecules that compose Pheromone A are heavier than the ones that compose Pheromone B. In this case Pheromone B is likely to spread through space faster than Pheromone A. Now, how does this spread area affects the colony's forage capability? The experiment in section \ref{sec:forage-radius-inv} investigates that. 

But pheromones are not the only way ants communicate, one of the other ways is by antennation. Ants are almost blind (((need citation))) and an effective way to 'see' each other is by using their antennas. As we will see later on, different types of ants have different chemical coats, and their fellow ants are able to recognise them, therefore the type of their fellows. This is can also be used as a way to communicate. For example, if a worker is following a pheromone trail that is supposed to lead to a source of food. But suddenly it starts to come across many workers moving in the direction of the nest, and most of these workers are not caring food. This could be interpreted as a sign of danger. If workers that are supposed to collect food, are running back to the nest without any, something ahead must be wrong. How does this second way of communication affects the velocity of reaction of the colony to danger? How does the variation of the number of workers going back to the nest without food as a threshold of danger situation affects how the colony reacts to danger? The experiment in section \ref{sec:warn-phero-inv} investigate that. 

In order to carry these experiments a flexible computational model that will enable studies on social multi-agent complex systems is proposed. This model needs to be extensible and loose enough to give users the freedom to create complex computational simulations. As in nature the model formalises a decentralised set of entities, now taking advantage of the tools computer models provide, the model also formalises an hierarchical infrastructure that can be used to describe different type of systems and track its components throughout simulations in order to acquire data.
 
In this model agents are defined by their type and a list of tasks associated to them. Inspired in the subsumption architecture proposed by \citeauthor{1087032}. \cite{1087032} \cite{Brooks1986b}. In the same way \citeauthor{1087032}'s robots are composed by reactive interconnected modules, in a way that effectively gives different priorities to different modules, in the computational model proposed by this project, agents are composed tasks, each task is completely independent from each other as far their execution goes. There is nothing to prevent a task to use another task to achieve a desired goal though. Differently from the subsumption architecture the agents from the proposed model are not purely reactive, they are able to reason what they should do next and then select which task to execute if they are programmed to do so. The users have the freedom to implement the agents as they wish. Simple BDI \cite{bratman1999intention} \cite{wooldridge2009introduction} agents can be implemented by extending the abstract infrastructure with minimum effort.

Agent-based modelling is at the core of the computational model. Even though more formal agent technology, such as communication protocols, are not used by the computational model, just the fact agents are autonomous, it offers the ideal toolset to be used to create decentralised complex systems.

The model is presented in UML as class diagrams and activity diagrams; it is implemented using the computer language Java. UML has become the standard modelling language within the OOP world as well as it has become a popular popular technique outside the OOP circle. \cite{fowler2004uml}, among other features it provides a set of diagrams that can be used to describe classes, objects, action sequences and so on. As far this project is concerned, class diagrams are the only part of the language used.

Java is a high level object-oriented programming language. It has been released publicly in 1995 and has developed to a powerful platform. Historically, Java has had great success in the enterprise environment, due its flexibility, robustness and an entire ecosystem created around the core language that allows users to concentrate on the important parts of their work, leaving the language frameworks to deal with low level issues such as transaction management. In recent years, Java has been increasingly adopted by users from the academic background, with the introduction of libraries like JScience and the development in hardware, it is possible to take advantage of features that are exclusive to java to facilitate researches in a variety of areas.

Mostly important for this project, Java offers a very comprehensive, high level, shared-memory management and threading facilities. If not coding low level libraries users do not need to use complex synchronisation techniques like semaphores, the language offers key words, context blocks and utility libraries ready to be used when multi-threading is required.

In this project each Agent is run as a isolated thread in any available CPU, each of agent completely autonomous from each other, they only share the same information about the environment around them, that is, there is not access to global information, only the local context is available to take decisions and execute tasks. The simulations executed for this project used from 10 up to 300 agents, there is no theoretical limit for the number of agents (that is threads) that could be used, only physical restrictions such as memory availability will impose an limit the number of agents or nodes used in the simulations.